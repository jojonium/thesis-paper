\section{Background and Related Work}
\label{sec:related-work}

Related work has been conducted in both of the main focus areas of this project:
the use of UI interaction data for security and implementing SDN on the Android
platform. We now review such work.

\subsection{UI Interaction and Security}
\label{sec:ui-interaction-and-security}

Previous work has investigated how user interactions with the UI can be used to
identify human-initiated behavior on various platforms. Harbinger examines user
interface activity on Windows to provide context for network requests to
previously unknown hosts in a default-deny environment~\cite{chuluundorj2019}.
This approach ``hooks into'' mouse and keyboard inputs using Microsoft's UI
Automation library~\cite{microsoft2018}, intercepting the input before it is
received by the intended application. Although Harbinger's operations are
performed synchronously they still only added six milliseconds or less of
latency to 96\% of flows. In contrast, the relevant library on Android is
asynchronous and event-driven~\cite{googledevelopers2020}.

Kwon \etal also used UI interaction data to distinguish human-generated from
automated network requests on Windows~\cite{kwon2011}. They propose a host-based
system for combating botnets by leveraging UI interaction data. But their
approach of labelling any flow initiated within one second of an interaction
with the flow's process as ``user generated'' ignores much of the context that
can be gained from UI data and makes deliberate evasion easy.

Most previous work on Android has used the platform's
\textit{AccessibilityService}, a library that includes the ability to respond to
UI changes among other accessibility features. AppIntent~\cite{yang2013} uses
this service to determine if an app is leaking private information by building a
graph of UI interactions that could lead to information leakage. However,
AppIntent is only designed to determine whether an app \textit{could} leak
private information through static analysis, and does not work in real time on a
flow-to-flow basis.

% SDN in general section
\subsection{Software-defined Networking}
\label{sec:software-defined-networking}

Software-defined networking (SDN) has been studied as a tool for giving network
administrators more information about, and control over, intra-network traffic.
In essence, SDN centralizes network intelligence by separating the forwarding of
packets (the data plane) from their routing (the control plane). In this
paradigm network switches merely forward packets, while all control and logic is
centralized with an SDN controller server~\cite{kim2013}.

SDN gives administrators more fine-grained control over packet routing rules,
which enables more sophisticated policy enforcement. It also provides insight
into potential issues and congestion in the network from one centralized control
point, rather than spread across many switches and routers. The main drawback of
this approach is the high overhead that causes it to scale
poorly~\cite{benzekki2016}.

Some authors have investigated using end nodes, instead of routers and switches,
as SDN rule caches to alleviate this problem~\cite{taylor2017, chuluundorj2019}.
In this approach, each host caches rules from the SDN controller and makes
routing decisions about its own network flows.

Others have investigated the potential for improving SDN's capabilities by
adding context~\cite{yang2015}, an approach which \textsc{Appjudicator} also
attempts. SDN rules could be more powerful and granular with more metadata, for
example a network flow's initiating application.

\subsection{The Android Phone as an SDN Agent}
\label{sec:the-android-phone-as-an-sdn-agent}

Many previous applications have used an Android smartphone as a SDN agent and
rule cache. Hong~\etal explores the tools available to organizations for
managing BYOD Android devices and proposes a system for applying SDN policies to
these phones~\cite{hong2016}. They were able to apply app-specific rules that
were aware of device context such as GPS location with minimal overhead.

While Hong~\etal \cite{hong2016} was aimed at corporate users, HanGuard
investigated using SDN principles to protect Internet of things (IoT) devices in
home networks~\cite{demetriou2017}. Their application provided tools for users
to easily define SDN rules for IoT apps, IoT devices, and users.

While each of these areas has been extensively researched alone, only
Chuluundorj~\cite{chuluundorj2019} and Kwon~\etal \cite{kwon2011} have
investigated how UI and network data can be combined for security purposes.
However, both of these applications were developed for the Windows
platform---bringing the idea to Android involves a new set of constraints and
challenges. For example, our proposed solution can be installed as a normal app,
and does not require recompiling the kernel or rooting the phone. This means
that we cannot modify the operating system kernel as the Windows solutions do,
and must abide by Android's strict permissions system.

\newpage
