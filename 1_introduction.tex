\section{Introduction}
\label{sec:introduction}

% Need

Current corporate network administration tools have little insight into the
connections made by employee-owned smartphones, especially in ``Bring Your Own
Device'' (BYOD) scenarios that are becoming increasingly common. It is difficult
for large organizations to maintain security across their networks with such
little knowledge about, and control over, these mobile devices. The fact that
these devices enter and exit the corporate network and connect to other
networks, such their cellular data provider's, on a daily basis complicates this
issue. Smartphones have a wide range of powerful networking and computational
abilities, and are typically privately owned by employees, which raises further
concerns when integrating them into secure network environments.

For example, smartphones have been targeted by malware specifically designed to
infiltrate corporate networks~\cite{kan2016}, such as ``Dresscode'', which
disguises itself as a legitimate app in order to steal data and add infected
devices to a botnet~\cite{palmer2016}. The ``xHelper'' malware can automatically
download and install arbitrary software specified by an attacker, and persists
even after a factory reset~\cite{vijayan2020}. A black market
malware-as-a-service model called Black Rose Lucy even offers control of
infected Android devices to paying customers, potentially giving any malicious
actor an entry point to a secure network~\cite{wong2018}.

If mobile devices are to have access to sensitive corporate network
infrastructure, either from home or from work, administrators need new tools to
monitor and control their network connections. These tools need to be powerful
enough to detect when network connections are made by Trojan horse malware that
is disguised as a benign app, while being still being easy to deploy and manage.
For example, some existing solutions require recompiling Android's kernel or
gaining root access to their device, which is impractical for most users and
exposes them to additional security risks~\cite{google2020}.

% Approach

To solve these issues on Android devices, we propose a new app called
\textsc{Appjudicator} which leverages user interface (UI) interaction and
software-defined networking (SDN) principles to determine whether network flows
are legitimately user-initiated. The app has two components: an accessibility
service that monitors UI interaction and a VPN service that intercepts network
flows and acts as an OpenFlow agent. We use these two sources of information to
correlate network flows with user interaction in a way that is difficult for
malware to evade, since it cannot physically interact with hardware input
devices.

The UI monitoring component will use Android's accessibility API to
asynchronously record physical hardware inputs, such as tapping or swiping the
touch screen. This gives administrators the ability to accurately distinguish
between human-driven and automated network requests.  Meanwhile, the VPN service
acts as an OpenFlow agent and rule cache, giving administrators fine-grained
control over which connections the device is allowed to make. Flows are
augmented with context from the specific device and application, and can be
elevated to the organization's SDN controller if necessary.

% Benefit over cost

Our application provides organizations with a new set of powerful, easy to use
tools for monitoring BYOD Android smartphones in a simple app package that is
easy for users to install. It does not require rooting, recompiling the kernel,
or any other cumbersome processes. It can distinguish between user-generated and
automated network requests with a high degree of confidence using techniques
that are difficult for malware to evade. This makes users safer by detecting
stealthy malware on their devices, and improves the organizations network
security.

The costs of our system include the resource overhead of running the app (CPU
cycles, battery power, etc.) and the cost of installing, running, and
maintaining the controller server. We expect these costs to be a relatively
small addition to existing resources. The app's UI monitoring and VPN service
also add some latency to network requests performed on the phone. In our
experiments \textsc{Appjudicator} added less than 200 milliseconds of total
latency to new TCP connections in 97\% of trials. UDP connections and subsequent
TCP requests to established connections have even less added latency.

% TODO can we justify this latency?

% Competition

Using UI data to distinguish user-initiated behavior and the use of a smartphone
as an SDN agent have been studied by others before, but \textsc{Appjudicator}
combines these aspects in new and important ways. Android host-based SDN agets
have been investigated by Hanguard~\cite{demetriou2017}, and using UI
interaction as context for identifying malicious app behavior was examined in
AppIntent~\cite{yang2013}, but our system combines these approaches to form a
novel solution for a distinct use case. One key difference is that AppIntent
uses machine learning to conduct static analysis of apps, while
\textsc{Appjudicator} enables live monitoring and response to malicious network
flows in real time.

Kwon~\etal have proposed a system that uses UI data to distinguish
user-generated network flows, and the combination of UI monitoring and SDN has
been implemented on Microsoft Windows by Harbinger~\cite{chuluundorj2019}.  We
solve new challenges by implementing this strategy on the Android platform. For
example, we are constrained by Android's rigid permissions system, we do not
have root access to the device, and we cannot modify the kernel. These
restrictions require novel techniques and systems.

Our work focuses on exploring the following research question: 
\begin{quote}
	\textit{Can UI interaction and network activity be used to successfully
		predict and associate network flows with user actions on Android devices
		with acceptable overhead?}
\end{quote}

To investigate we perform a study with several popular Google Play apps to
determine whether \textsc{Appjudicator} can successfully detect and block
suspicious non-user-initiated network requests. % TODO insert results
We also measure the total added latency of the VPN and SDN agent, finding that
our system adds less than 200 milliseconds of latency in 97\% of new TCP
connections.
% TODO add more results

\newpage
