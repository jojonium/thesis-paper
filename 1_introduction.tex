\section{Introduction}
\label{sec:introduction}

% Need

Many employees are now using their own personal smartphones for work purposes,
as working from home becomes widespread and companies adopt ``Bring Your Own
Device'' (BYOD) policies. Existing corporate network administration tools have
little insight into the connections made by employee-owned smartphones. It is
difficult for large organizations to maintain security across their networks
with such little knowledge about, and control over, these mobile devices. The
fact that these devices enter and exit the corporate network and connect to
other networks, such as their cellular data provider's, on a daily basis
complicates this issue.

Smartphones have a wide range of powerful networking and computational
abilities, and are typically privately owned by employees, which raises further
concerns when integrating them into secure network environments. For example,
smartphones have been targeted by malware specifically designed to infiltrate
corporate networks~\cite{kan2016}, such as ``Dresscode'', which disguises itself
as a legitimate app in order to steal data and add infected devices to a
botnet~\cite{palmer2016}. The ``xHelper'' malware targets phones used for both
personal and professional purposes; can automatically download and install
arbitrary software specified by an attacker; and persists even after a factory
reset~\cite{vijayan2020}. A black market malware-as-a-service model called
``Black Rose Lucy'' even offers control of infected Android devices to paying
customers, potentially giving any malicious actor an entry point to a secure
network~\cite{wong2018}.

If mobile devices are to have access to sensitive corporate network
infrastructure, either from home or from work, administrators need new tools to
monitor and control their network connections. These tools could provide the
information needed to monitor network activity for malware, provision network
resources based on which apps are being used, or debug complex distributed
applications. However, if the tools are not easy to deploy and manage, it will
be difficult to convince end users to adopt them. For example, similar
solutions on Windows~\cite{chuluundorj2019, kwon2011} require administrator
privileges or the ability to modify the kernel. This is impractical for most
users to do on a smartphone and exposes them to additional security
risks~\cite{google2020}.

% Approach

To solve these issues on Android devices, we propose a new app called
\textsc{Appjudicator} that leverages user interface (UI) interaction and
software-defined networking (SDN) principles to determine whether network flows
are legitimately user-initiated. The app has two components: an accessibility
service that monitors UI interaction and a VPN service that intercepts network
flows. We use these two sources of information to correlate network flows with
user interaction, and we provide this information to network administrators by
acting as a host-based SDN agent.

The UI monitoring component uses Android's accessibility API to asynchronously
record physical hardware inputs, such as tapping or swiping the touch screen.
This gives administrators the ability to accurately distinguish between
human-driven and automated network requests. Meanwhile, the VPN service acts as
an OpenFlow agent and rule cache, giving administrators fine-grained control
over which connections the device is allowed to make. Flows are augmented with
context from the specific device and application, and can be elevated to the
organization's SDN controller along with UI context data if necessary.

% Benefit over cost

Our application provides organizations with a new set of powerful, easy-to-use
tools for monitoring BYOD Android smartphones in a simple app package that is
easy for users to install. It does not require rooting, recompiling the kernel,
or any other cumbersome processes. It can distinguish between user-generated and
automated network requests with a high degree of confidence using techniques
that are difficult for malware to evade. This makes users safer by detecting
stealthy malware on their devices and improves the organization's network
security.

The costs of our system include the resource overhead of running the app (CPU
cycles, battery power, etc.) and the expense of installing, running, and
maintaining the controller server. We found these costs to be a relatively
small addition to existing resources. The app's UI monitoring and VPN service
also add some latency to network requests performed on the phone. In our
experiments, \textsc{Appjudicator} added less than 14 milliseconds of total
latency to 95\% of packets processed by the app.

% Competition

Using UI data to distinguish user-initiated behavior and using a smartphone as
an SDN agent have previously been studied, but \textsc{Appjudicator} combines
these aspects in new and important ways. Android host-based SDN agets have been
investigated by Hanguard~\cite{demetriou2017}, and using UI interaction as
context for identifying malicious app behavior was examined in
AppIntent~\cite{yang2013}, but our system combines these approaches to form a
novel solution for a distinct use case. One key difference is that AppIntent
uses machine learning to conduct static analysis of apps, while
\textsc{Appjudicator} enables live monitoring and response to malicious network
flows in real time.

Kwon~\etal have proposed a system that uses UI data to distinguish
user-generated network flows, and the combination of UI monitoring and SDN has
been implemented on Microsoft Windows by Harbinger~\cite{chuluundorj2019}.  We
solve new challenges by implementing this strategy on the Android platform. For
example, since we are constrained by Android's rigid permissions system, we do
not have root access to the device and cannot modify the kernel. These
restrictions require novel techniques and systems.

This work focuses on implementing a proof of concept networking tool for
Android. We explore the following research question:
\begin{quote}
	\textit{Can UI interaction and network activity successfully be used to 
		predict and associate network flows with user actions on Android devices
		with acceptable overhead?}
\end{quote}
We leave the potential applications of this data for network security to future
work, instead examining the effectiveness of our strategy for differentiating
user-initiated flows and measuring the system's CPU, memory, and energy costs.

To investigate, we perform a study with several real apps to determine whether
\textsc{Appjudicator} can successfully detect user-initiated network requests.
We found that even with a simple timing-based strategy the app could
successfully associate network flows with the UI interaction that initiated
them. We also measure the total added latency of the VPN and SDN agent, finding
that our system adds less than 14 milliseconds of total end-to-end latency to
95\% of intercepted packets. On a standard 60 Hz screen, this latency is
imperceptible to users.

In summary, this paper makes the following contributions:

\begin{itemize}
	\item We propose a novel system for associating network flows with user
		interface context on Android. Unlike existing solutions, our application
		implements a host-based SDN agent and does not require root access to
		the device.
	\item We design and implement a practical prototype of the system on
		Android, called \textsc{Appjudicator}, which integrates with other SDN
		infrastructure using a subset of the OpenFlow protocol.
	\item We evaluate the performance and efficacy of \textsc{Appjudicator} on
		real and virtual Android devices with real-world apps. The results show
		that this prototype has a low impact on CPU, memory, and battery usage,
		and adds a minimal amount of total latency to network communications
		while achieving its goal of differentiating user-initiated flows and
		associating flows with UI context.
\end{itemize}

\newpage
