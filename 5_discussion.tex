\section{Discussion}
\label{sec:discussion}

Here we discuss the practical applications of our work, as well as its
limitations, future work, and possible improvements. We also use the results of
our work to answer our original research question.

\subsection{Implications and Applications}
\label{sec:implications-and-applications}

\subsection{Future Work}
\label{sec:future-work}

More work is needed to implement \textsc{Appjudicator} as a practical network
security tool in an enterprise setting. Our work's biggest limitation is that it
is only available for Android. Android is the leading smartphone operating
system, capturing over 71\% of the global market share,~\cite{statcounter2021}
but implementing a similar system on iOS (which makes up virtually all of the
remaining market share) would make the system accessible to all BYOD employees.
This work would be necessary before a company could require its employees to use
\textsc{Appjudicator}---extra insight and control over \textit{only} Android
users would provide little benefit.

Custom SDN controller software could also be written to take advantage of
\textsc{Appjudicator}'s enhanced context information. Our app sends metadata
about recent UI interactions along with each \texttt{packet\_in}, so a custom
controller could take advantage of this extra context in its decision-making
process. For example, a controller could have a more sophisticated algorithm for
inferring user intent, and use this to block non-user-initiated flows from
mobile devices. A custom controller could also empower system administrators to
write fine-grained, context-aware SDN policies. 

Further work could also be done to mitigate some of the limitations discussed in
Section~\ref{sec:limitations}. Because our work is a proof of concept little
development time was spent on optimization or graphical polish. A more
user-friendly UI and settings menu could be implemented for the app. Our latency
test results (described in Section~\ref{sec:latency-test-results}) indicate that
the app adds less than 10 milliseconds of total latency to 95\% of processed
packets, but this could probably be reduced with further performance
optimizations.

\subsection{Limitations}
\label{sec:limitations}

\subsection{Conclusion}
\label{sec:conclusion}

\newpage

