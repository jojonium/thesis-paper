\section{Discussion}
\label{sec:discussion}

Here we discuss the practical applications of our work, as well as its
limitations, future work, and possible improvements. We also use the results of
our work to answer our original research question.

\subsection{Implications and Applications}
\label{sec:implications-and-applications}

\textsc{Appjudicator} was designed for an enterprise BYOD setting, as a tool to
give system administrators more control over employee-owned smartphones.  Our
app would work with an OpenFlow SDN controller to enforce company policy rules.
With some modification, \textsc{Appjudicator} could also be used in other
applications. For example, our app could adapted for use by personal users in
residential networks, connected to a cloud-based SDN controller like the one
proposed by Taylor~\etal\cite{taylor2017shue}

The use of \textsc{Appjudicator} in an enterprise network could raise some
privacy concerns for end users. Users may object to having logs of every UI
interaction they make on their personal device sent to their employer. The
system could potentially expose private data to the SDN
controller,~\cite{kalysch2018} so administrators will have to carefully consider
how their policy rules impact user privacy. A potential solution to this could
be to define a set of general user intent profiles on the device, and only send
the closest matching profile to the server with a \texttt{packet\_in} rather
than the entire UI interaction metadata.

\subsection{Future Work}
\label{sec:future-work}

More work is needed to implement \textsc{Appjudicator} as a practical network
security tool in an enterprise setting. Our work's biggest limitation is that it
is only available for Android. Android is the leading smartphone operating
system, capturing over 71\% of the global market share,~\cite{statcounter2021}
but implementing a similar system on iOS (which makes up virtually all of the
remaining market share) would make the system accessible to all BYOD employees.
This work would be necessary before a company could require its employees to use
\textsc{Appjudicator}---extra insight and control over \textit{only} Android
users would provide little benefit.

Custom SDN controller software could also be written to take advantage of
\textsc{Appjudicator}'s enhanced context information. Our app sends metadata
about recent UI interactions along with each \texttt{packet\_in}, so a custom
controller could take advantage of this extra context in its decision-making
process. For example, a controller could have a more sophisticated algorithm for
inferring user intent, and use this to block non-user-initiated flows from
mobile devices. A custom controller could also empower system administrators to
write fine-grained, context-aware SDN policies. 

Further work could also be done to mitigate some of the limitations discussed in
Section~\ref{sec:limitations}. Because our work is a proof of concept, little
development time was spent on optimization or graphical polish. A more
user-friendly UI and settings menu could be implemented for the app. Our latency
test results (described in Section~\ref{sec:latency-test-results}) indicate that
the app adds less than 10 milliseconds of total latency to 95\% of processed
packets, but this could probably be reduced with further performance
optimizations.

Some researchers have investigated SDN architectures that involve multiple
distributed controller servers.~\cite{dixit2013, oktian2017}
\textsc{Appjudicator} currently connects to only one statically-defined
controller server, but could be extended in future work to connect to several.

\subsection{Limitations}
\label{sec:limitations}

For ease of development and testing, \textsc{Appjudicator} only supports IPv4,
and can only handle UDP and TCP as transport layer protocols. This is enough to
power the vast majority of everyday networking applications, but support could
be added for IPv6 and other transport layer protocols.

Our strategy for defining user-generated flows can give a rough estimate of
whether a user is interacting with a particular app, but it is far from perfect.
A flow from an application is considered user-initiated if a user made any UI
interaction with that app in the past two seconds, so malicious network flows
could be allowed if they happen to be sent within this time period. A malicious
app with knowledge of this system could also generate fake accessibility events
to give the impression that a user is interacting with an app. To mitigate this
issue, future work could focus on improving how we define a flow as
user-initiated. For example, a system for detecting user intent similar to the
one presented by Shirley and Evans~\cite{shirley2008} could be added to
\textsc{Appjudicator}. A system like this would make it easier to detect whether
whether the accessibility events we receive are legitimate.

\subsection{Conclusion}
\label{sec:conclusion}

In this thesis, we implemented a host-based SDN agent and UI monitoring system
on Android. We demonstrated that these systems can be combined to collect
network data and UI context, and this data can be used to successfully
distinguish which flows on the device are legitimately user initiated. Our
experiments showed that the system can run in the background with minimal
overhead to CPU, memory, and battery usage, and adding an acceptably low amount
of overall end-to-end latency to network operations on the device. Finally, we
discussed how the application can be used as a network security tool in
enterprise environments and what future work is required to turn our proof of
concept into a practical application.

\newpage

