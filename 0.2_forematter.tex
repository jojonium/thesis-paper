\pagenumbering{roman}

\begin{abstract}
	Smartphones are becoming increasingly important in all aspects of life,
	including corporate environments, where ``bring your own device'' (BYOD)
	policies are gaining widespread acceptance. Malware already exists to take
	advantage of Android phones in BYOD settings, aiming to take control of
	devices with access to privileged information by disguising itself as a
	benign app. These stealthy malware could be easier to detect if network
	administrators had more insight into employee-owned smartphones. We propose
	a system, called \textsc{Appjudicator}, to address this issue. It implements
	an accessibility service to monitor user interactions with the UI of other
	apps, so this context can be used in malware detection. For example, if an
	app sends a new network request without any user interaction, this flow
	could be the result of malware and should be investigated more. Our app is a
	host-based software defined networking (SDN) agent, and works in conjunction
	with an SDN controller to monitor and control the phone's networking
	abilities based on the organizations SDN rules and our UI context. We build
	a proof of concept application and find that it can successfully combine
	network and UI data with only 10 milliseconds of total added latency in 95\%
	of flows.
\end{abstract}

\clearpage

\section*{Acknowledgments}
\label{sec:acknowledgments}

I would like to express my gratitude to my advisor, Professor Craig Shue, for
guiding me through my graduate education with helpful advice and insightful
feedback. Without his patience and wisdom this thesis could not have happened. I
would also like to thank Yu Liu for his technical help and support throughout
the project. Additionally, I want to express my appreciation to my thesis
reader, Professor Robert Walls, for his helpful feedback on this paper. His
questions and criticism, along with every member of the WPI Cake Lab, are what
turned this paper from a very rough summary to a polished final product.

\clearpage

\tableofcontents

\clearpage
\pagenumbering{arabic}
